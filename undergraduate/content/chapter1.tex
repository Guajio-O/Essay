%!TEX root = ../../csuthesis_main.tex
\chapter{引言}

\section{研究背景与意义}

随着交通智能化和城市化进程的加快,复杂交通环境中行人安全和交通效率问题日益显现。行人导航和控制作为智慧交通系统的重要组成部分,其优化对提升整体交通运行效率、减少交通事故具有重要意义。然而,现阶段的行人控制系统在动态复杂场景下仍面临诸多挑战,例如实时路径规划、动态避障、多智能体协作等问题。

近年来,虚幻引擎(Unreal Engine)以其高度真实的场景渲染和开放的接口,被广泛应用于自动驾驶与交通仿真研究中。基于虚幻引擎的CARLA仿真平台提供了高度可定制的交通模拟环境,涵盖了复杂的城市街道布局、多天气条件和动态交通参与者等,为行人导航与控制技术的研究提供了高效支持。同时,深度学习和强化学习技术的迅猛发展为行人行为建模和优化提供了新工具。例如,基于深度强化学习的行人路径规划和避障策略可以实现复杂动态环境下的智能决策,并有效提升行人安全性和导航效率。

本研究基于CARLA仿真平台,结合深度学习和强化学习技术,探索复杂动态环境中行人导航与控制的优化方法,旨在提升动态环境中的行人导航效率与安全性,并为智慧交通系统的设计提供理论支持。

\section{国内外研究现状}

\subsection{国内研究现状}

国内学者在交通仿真与行人导航领域的研究逐步深入,广泛应用深度学习和强化学习技术优化交通信号和行人路径规划。例如,基于YOLOv5的多目标检测研究中,利用CARLA仿真平台生成多样化交通场景和数据集,实现了对行人和车辆的高效检测和行为分析。此外,国内一些研究通过结合深度强化学习技术优化行人信号控制策略,例如使用优先经验回放机制的深度Q网络(DQN),显著减少行人和车辆的延误。

然而,现有研究多集中于单一场景的优化,缺乏对复杂多变动态环境下行人行为建模和优化的系统研究。特别是,对于如何在高密度人群和动态障碍的复杂环境中实现高效、安全的行人导航,仍然需要进一步探索。

\subsection{国外研究现状}

国外学者较早将虚拟仿真技术与强化学习技术结合应用于交通优化研究。例如,国外研究利用CARLA平台构建了多样化的交通场景,通过改进的YOLOv8算法优化行人检测精度和鲁棒性。改进后的YOLOv8通过引入UniRepLKNet骨干网络和Shape-IoU损失函数,在复杂场景中显著提升了行人检测性能。

此外,国外研究还聚焦于多智能体强化学习框架下的行人导航优化,采用策略优化(PPO)和双深度Q网络(DDQN)等方法,实现了复杂动态环境中多智能体的协同控制。尽管已有研究在复杂交通环境中取得了较好的优化效果,但在高动态性、多场景化的仿真环境中如何平衡安全性与效率,仍是一个亟待解决的关键问题。

\section{研究目标}

本研究的主要目标如下:

利用CARLA仿真平台构建多样化的复杂交通场景,模拟真实的行人导航环境;

基于深度强化学习技术,设计高效的行人导航控制算法,解决复杂动态环境中的路径规划与避障问题;

验证所提算法在仿真环境中的性能,并分析其在复杂动态场景中的适用性和优化效果。

\section{研究内容}

为实现上述目标,本文的研究内容主要包括:

复杂动态环境下行人导航问题分析:总结复杂动态场景对行人导航的影响因素,分析现有技术的不足。

基于深度强化学习的行人导航算法设计:构建状态空间、动作空间和奖励函数,设计动态环境中的行人导航策略。

CARLA仿真平台的场景构建:模拟多样化的交通场景,包括城市街道、交叉路口和高密度人群环境。

实验验证与性能评估:验证所提算法在不同场景中的性能,并与传统方法进行对比分析。

